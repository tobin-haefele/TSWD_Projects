% Options for packages loaded elsewhere
\PassOptionsToPackage{unicode}{hyperref}
\PassOptionsToPackage{hyphens}{url}
%
\documentclass[
]{article}
\usepackage{amsmath,amssymb}
\usepackage{iftex}
\ifPDFTeX
  \usepackage[T1]{fontenc}
  \usepackage[utf8]{inputenc}
  \usepackage{textcomp} % provide euro and other symbols
\else % if luatex or xetex
  \usepackage{unicode-math} % this also loads fontspec
  \defaultfontfeatures{Scale=MatchLowercase}
  \defaultfontfeatures[\rmfamily]{Ligatures=TeX,Scale=1}
\fi
\usepackage{lmodern}
\ifPDFTeX\else
  % xetex/luatex font selection
\fi
% Use upquote if available, for straight quotes in verbatim environments
\IfFileExists{upquote.sty}{\usepackage{upquote}}{}
\IfFileExists{microtype.sty}{% use microtype if available
  \usepackage[]{microtype}
  \UseMicrotypeSet[protrusion]{basicmath} % disable protrusion for tt fonts
}{}
\makeatletter
\@ifundefined{KOMAClassName}{% if non-KOMA class
  \IfFileExists{parskip.sty}{%
    \usepackage{parskip}
  }{% else
    \setlength{\parindent}{0pt}
    \setlength{\parskip}{6pt plus 2pt minus 1pt}}
}{% if KOMA class
  \KOMAoptions{parskip=half}}
\makeatother
\usepackage{xcolor}
\usepackage[margin=1in]{geometry}
\usepackage{longtable,booktabs,array}
\usepackage{calc} % for calculating minipage widths
% Correct order of tables after \paragraph or \subparagraph
\usepackage{etoolbox}
\makeatletter
\patchcmd\longtable{\par}{\if@noskipsec\mbox{}\fi\par}{}{}
\makeatother
% Allow footnotes in longtable head/foot
\IfFileExists{footnotehyper.sty}{\usepackage{footnotehyper}}{\usepackage{footnote}}
\makesavenoteenv{longtable}
\usepackage{graphicx}
\makeatletter
\def\maxwidth{\ifdim\Gin@nat@width>\linewidth\linewidth\else\Gin@nat@width\fi}
\def\maxheight{\ifdim\Gin@nat@height>\textheight\textheight\else\Gin@nat@height\fi}
\makeatother
% Scale images if necessary, so that they will not overflow the page
% margins by default, and it is still possible to overwrite the defaults
% using explicit options in \includegraphics[width, height, ...]{}
\setkeys{Gin}{width=\maxwidth,height=\maxheight,keepaspectratio}
% Set default figure placement to htbp
\makeatletter
\def\fps@figure{htbp}
\makeatother
\setlength{\emergencystretch}{3em} % prevent overfull lines
\providecommand{\tightlist}{%
  \setlength{\itemsep}{0pt}\setlength{\parskip}{0pt}}
\setcounter{secnumdepth}{-\maxdimen} % remove section numbering
\ifLuaTeX
  \usepackage{selnolig}  % disable illegal ligatures
\fi
\IfFileExists{bookmark.sty}{\usepackage{bookmark}}{\usepackage{hyperref}}
\IfFileExists{xurl.sty}{\usepackage{xurl}}{} % add URL line breaks if available
\urlstyle{same}
\hypersetup{
  pdftitle={Unhoused Funding Data Report},
  pdfauthor={Tobin Haefele},
  hidelinks,
  pdfcreator={LaTeX via pandoc}}

\title{Unhoused Funding Data Report}
\author{Tobin Haefele}
\date{}

\begin{document}
\maketitle

{
\setcounter{tocdepth}{2}
\tableofcontents
}
\subsection{Introduction}\label{introduction}

In this report we partnered with the city of Missoula's ``Community
Planning, Development, and Innovation'' (CPDI) department, specifically
with the Community Development Division. In an effort to understand and
monitor Missoula's unhoused population the CPDI department joined the
Homeless Management Information System (HMIS) data collection effort by
establishing a coordinated entry system. This allowed them to build
datasets that can tell us much about the unhoused population. Along with
this they have also executed a survey of the unhoused population to
provide further data for us. Thanks to their hard work we have data to
analyze and report on.

\subsection{Executive summary}\label{executive-summary}

This report presents an in-depth analysis of the CPDI department's data,
yielding several key insights:

Spending Distribution of funds was primaraly spent of prevention of
homelessness rather than on those who are currently homeless. This is in
line with the CPDI's goal of preventing homelessness before it happens.
The majority of funds are going towards rent which aligns with the goal
of preventing homelessness before it happens.

The funding sources allocation found that the majority of funding is
coming from three main sources, the AHTF (Affordable Housing Trust Fund
(City)), WF (Wells Fargo). The AHTF had the highest average spend per
individual at \$559, followed by WF (Wells Fargo) at \$529. The EFSG had
the highest average spend per individual at \$762 but this could be
contributed to the small sample size of 10 individuals.

Finally our demographic analysis found that the majority of funding is
going towards the White demographic at 84\%. This is in contrast to the
general unhoused population within the MCES, where the percentage of
White individuals is significantly lower at 67\%. Further exploration of
this disparity highlights potential inequities in funding allocation
across minority demographic groups.

These findings offer valuable insights into the utilization and
distribution of resources, underscoring the need for further exploration
and potential adjustments to ensure more equitable access and targeted
support for diverse demographic groups.

\subsubsection{Analyzing the utilization of the housing solutions fund
by prevention or currently
homeless}\label{analyzing-the-utilization-of-the-housing-solutions-fund-by-prevention-or-currently-homeless}

Our first analysis is how the housing solutions fund is being utilized
to help the unhoused population. We will look at the total amounnt of
money spent on prevention and the total amount spent on those who are
currently homeless. Along with this we also want to look at what the
funding is being used for and how much is being used in each category.

Initial analysis of the housing solutions fund found that a total of
\$193,628 was put towards either prevention or assistance to those
currently unhoused. Of this total around 77\% of the funding is being
used for prevention of homelessness compared to 23\% being allocated for
those who are currently homeless. This disparity could be explained by
the fact that it is much cheaper to prevent homelessness than it is to
assist those who are currently homeless. This also aligns with the
CPDI's goal of preventing homelessness before it happens.

\begin{verbatim}
## Warning: The `<scale>` argument of `guides()` cannot be `FALSE`. Use "none" instead as
## of ggplot2 3.3.4.
## This warning is displayed once every 8 hours.
## Call `lifecycle::last_lifecycle_warnings()` to see where this warning was
## generated.
\end{verbatim}

\includegraphics[width=0.8\linewidth]{report_files/figure-latex/unnamed-chunk-1-1}

In this section I want to take a look at funding and how much is being
spent on each category. This will allow us to view what the funding is
being used for and how much is being spent on each category, allowing us
to see if the funding is being used in the most optimal way.

Within the Prevention group we can see that the over 70\% of funding is
going towards rent while deposits make up around 14\%. The makeup of
these categories make sense as preventing eviction or helping with
deposits to secure a place to live is a great way to prevent
homelessness.

Among those currently homeless the majority of funding is going toward
deposits (28\%) and transportation (23\%), both of these categories
support the idea of helping those who are homeless either get into a
home or get somewhere they have family/friends or a better opportunity.

\begin{verbatim}
## Warning in colnames(prevention_table)[1:3] <- new_column_names: number of items
## to replace is not a multiple of replacement length
\end{verbatim}

\begin{verbatim}
## Warning in colnames(literally_homeless_table)[1:3] <- new_column_names: number
## of items to replace is not a multiple of replacement length
\end{verbatim}

\begin{longtable}[]{@{}
  >{\raggedright\arraybackslash}p{(\columnwidth - 6\tabcolsep) * \real{0.4459}}
  >{\raggedright\arraybackslash}p{(\columnwidth - 6\tabcolsep) * \real{0.2027}}
  >{\raggedright\arraybackslash}p{(\columnwidth - 6\tabcolsep) * \real{0.1622}}
  >{\raggedright\arraybackslash}p{(\columnwidth - 6\tabcolsep) * \real{0.1892}}@{}}
\caption{Breakdown of Prevention Funding by Category}\tabularnewline
\toprule\noalign{}
\begin{minipage}[b]{\linewidth}\raggedright
Prevention or Literally Homeless
\end{minipage} & \begin{minipage}[b]{\linewidth}\raggedright
Category
\end{minipage} & \begin{minipage}[b]{\linewidth}\raggedright
Total Spent
\end{minipage} & \begin{minipage}[b]{\linewidth}\raggedright
percent\_total
\end{minipage} \\
\midrule\noalign{}
\endfirsthead
\toprule\noalign{}
\begin{minipage}[b]{\linewidth}\raggedright
Prevention or Literally Homeless
\end{minipage} & \begin{minipage}[b]{\linewidth}\raggedright
Category
\end{minipage} & \begin{minipage}[b]{\linewidth}\raggedright
Total Spent
\end{minipage} & \begin{minipage}[b]{\linewidth}\raggedright
percent\_total
\end{minipage} \\
\midrule\noalign{}
\endhead
\bottomrule\noalign{}
\endlastfoot
Prevention & Rent & \$105,461 & 70.857\% \\
Prevention & Deposits & \$21,335 & 14.335\% \\
Prevention & Other & \$16,103 & 10.819\% \\
Prevention & Utilities & \$5,266 & 3.538\% \\
Prevention & Applications & \$372 & 0.250\% \\
Prevention & Transportation & \$300 & 0.202\% \\
\end{longtable}

\begin{longtable}[]{@{}
  >{\raggedright\arraybackslash}p{(\columnwidth - 6\tabcolsep) * \real{0.4459}}
  >{\raggedright\arraybackslash}p{(\columnwidth - 6\tabcolsep) * \real{0.2027}}
  >{\raggedright\arraybackslash}p{(\columnwidth - 6\tabcolsep) * \real{0.1622}}
  >{\raggedright\arraybackslash}p{(\columnwidth - 6\tabcolsep) * \real{0.1892}}@{}}
\caption{Breakdown of Thoe Currently Homeless by
Category}\tabularnewline
\toprule\noalign{}
\begin{minipage}[b]{\linewidth}\raggedright
Prevention or Literally Homeless
\end{minipage} & \begin{minipage}[b]{\linewidth}\raggedright
Category
\end{minipage} & \begin{minipage}[b]{\linewidth}\raggedright
Total Spent
\end{minipage} & \begin{minipage}[b]{\linewidth}\raggedright
percent\_total
\end{minipage} \\
\midrule\noalign{}
\endfirsthead
\toprule\noalign{}
\begin{minipage}[b]{\linewidth}\raggedright
Prevention or Literally Homeless
\end{minipage} & \begin{minipage}[b]{\linewidth}\raggedright
Category
\end{minipage} & \begin{minipage}[b]{\linewidth}\raggedright
Total Spent
\end{minipage} & \begin{minipage}[b]{\linewidth}\raggedright
percent\_total
\end{minipage} \\
\midrule\noalign{}
\endhead
\bottomrule\noalign{}
\endlastfoot
Literally Homeless & Deposits & \$12,931.58 & 28.87\% \\
Literally Homeless & Transportation & \$10,322.75 & 23.05\% \\
Literally Homeless & Other & \$7,815.72 & 17.45\% \\
Literally Homeless & Utilities & \$6,185.57 & 13.81\% \\
Literally Homeless & Rent & \$5,938.60 & 13.26\% \\
Literally Homeless & Applications & \$1,597.39 & 3.57\% \\
\end{longtable}

\includegraphics[width=0.8\linewidth]{report_files/figure-latex/unnamed-chunk-2-1}

\subsubsection{Evaluating the sources of funding and their impact on
services
provided}\label{evaluating-the-sources-of-funding-and-their-impact-on-services-provided}

In this next section we will take a look at where this funding is coming
in from and how much is coming from each source. Using this data we can
identify which sources are being utilized the most and which are being
utilized the least.

Looking at the initial table we can see that just over half of the
funding is coming from AHTF (Affordable Housing Trust Fund (City)) which
is double the next closest source of funding OBT (Otto Bremer Trust) at
22\%. Despite the EFSG (Emergency Solutions Grant (Human Resource
Council program)) having a higher average spend on each individual at
\$762 per individual, this could be contributed to having a very small
sample size of 10 individuals. The highest average spend on each
individual when accounting for sample size ends up being the AHTF and WF
(Wells Fargo) sources, at \$559 and \$529 respectively.

\begin{longtable}[]{@{}
  >{\raggedright\arraybackslash}p{(\columnwidth - 6\tabcolsep) * \real{0.2532}}
  >{\raggedright\arraybackslash}p{(\columnwidth - 6\tabcolsep) * \real{0.1392}}
  >{\raggedright\arraybackslash}p{(\columnwidth - 6\tabcolsep) * \real{0.2152}}
  >{\raggedleft\arraybackslash}p{(\columnwidth - 6\tabcolsep) * \real{0.3924}}@{}}
\caption{Overall Breakdown of Where Funding is Coming
From}\tabularnewline
\toprule\noalign{}
\begin{minipage}[b]{\linewidth}\raggedright
Funding Source
\end{minipage} & \begin{minipage}[b]{\linewidth}\raggedright
Total Paid
\end{minipage} & \begin{minipage}[b]{\linewidth}\raggedright
Percent of Total
\end{minipage} & \begin{minipage}[b]{\linewidth}\raggedleft
Number of Individuals Assisted
\end{minipage} \\
\midrule\noalign{}
\endfirsthead
\toprule\noalign{}
\begin{minipage}[b]{\linewidth}\raggedright
Funding Source
\end{minipage} & \begin{minipage}[b]{\linewidth}\raggedright
Total Paid
\end{minipage} & \begin{minipage}[b]{\linewidth}\raggedright
Percent of Total
\end{minipage} & \begin{minipage}[b]{\linewidth}\raggedleft
Number of Individuals Assisted
\end{minipage} \\
\midrule\noalign{}
\endhead
\bottomrule\noalign{}
\endlastfoot
AHTF & \$98,248.10 & 50.74\% & 170 \\
EFSG & \$7,624.33 & 3.94\% & 10 \\
Home Coalition & \$5,577.37 & 2.88\% & 16 \\
Missoula Gives & \$1,455.00 & 0.75\% & 3 \\
OBT & \$43,069.69 & 22.24\% & 88 \\
Other sources/gifts & \$5,307.53 & 2.74\% & 17 \\
Providence & \$8,965.55 & 4.63\% & 24 \\
WF & \$23,381.03 & 12.08\% & 42 \\
\end{longtable}

\includegraphics[width=0.8\linewidth]{report_files/figure-latex/Source Evaluation-1}

Taking a look at the categories in which the majority of funding goes
for each source also sheds some insight into the goal for each source.
Across the board rental assistance seems to be the biggest category for
the majority of funding sources out there. Providence had the lowest
amount ratio of funds going to rent at 33\%, private donations was the
next lowest at 15\% and lastly Home Coalition was at 6\% before their
dissolution. The fact that the rest of the funding sources had over half
of their funding going to rent lends support to the goal of preventing
homelessness before it happens. The largest donor AHTF spent almost 80\%
of total donations on rent/deposits.

\includegraphics[width=0.8\linewidth]{report_files/figure-latex/unnamed-chunk-3-1}

In this last section I want to focus on the agencies/programs that refer
individuals to the Housing Solutions Fund. It's apparent that HRM has
the largest spread across each funding source, each of the major funding
sources AHTF, OBT, and WF contributed to each of the top 5 referral
forms. With more data I think it would be beneficiel to perform a basic
analysis to see if there is a correlation between the amount of funding
and the number of referrals from each agency/program. This could help us
identify which agencies/programs are the most effective at referring
individuals to the Housing Solutions Fund.

\includegraphics[width=0.8\linewidth]{report_files/figure-latex/unnamed-chunk-4-1}

\subsubsection{Comparing demographics of those who are receiving funding
to the overall population of those who are
unhoused}\label{comparing-demographics-of-those-who-are-receiving-funding-to-the-overall-population-of-those-who-are-unhoused}

An examination of the statistics reveals a notable trend in funding
distribution, with an average of 84\% allocated to individuals within
the White demographic. This distribution is in contrast to the general
unhoused population within the MCES, where the percentage of White
individuals is significantly lower at 67\%. Further exploration of this
disparity highlights potential inequities in funding allocation across
different demographic groups.

\begin{longtable}[]{@{}
  >{\raggedright\arraybackslash}p{(\columnwidth - 8\tabcolsep) * \real{0.3566}}
  >{\raggedleft\arraybackslash}p{(\columnwidth - 8\tabcolsep) * \real{0.1705}}
  >{\raggedright\arraybackslash}p{(\columnwidth - 8\tabcolsep) * \real{0.0853}}
  >{\raggedright\arraybackslash}p{(\columnwidth - 8\tabcolsep) * \real{0.2171}}
  >{\raggedright\arraybackslash}p{(\columnwidth - 8\tabcolsep) * \real{0.1705}}@{}}
\caption{Breakdown of Housing Solutions Fund Distribution by
Race/Ethnicity}\tabularnewline
\toprule\noalign{}
\begin{minipage}[b]{\linewidth}\raggedright
Race/Ethnicity
\end{minipage} & \begin{minipage}[b]{\linewidth}\raggedleft
Number of Individuals
\end{minipage} & \begin{minipage}[b]{\linewidth}\raggedright
Total Paid
\end{minipage} & \begin{minipage}[b]{\linewidth}\raggedright
Average Paid per Individual
\end{minipage} & \begin{minipage}[b]{\linewidth}\raggedright
Percent of Total Paid
\end{minipage} \\
\midrule\noalign{}
\endfirsthead
\toprule\noalign{}
\begin{minipage}[b]{\linewidth}\raggedright
Race/Ethnicity
\end{minipage} & \begin{minipage}[b]{\linewidth}\raggedleft
Number of Individuals
\end{minipage} & \begin{minipage}[b]{\linewidth}\raggedright
Total Paid
\end{minipage} & \begin{minipage}[b]{\linewidth}\raggedright
Average Paid per Individual
\end{minipage} & \begin{minipage}[b]{\linewidth}\raggedright
Percent of Total Paid
\end{minipage} \\
\midrule\noalign{}
\endhead
\bottomrule\noalign{}
\endlastfoot
White & 139 & \$85,836.62 & \$617.53 & 84.3\% \\
American Indian, Alaska Native, or Indigenous & 17 & \$10,743.93 &
\$632.00 & 10.5\% \\
Black, African American, or African & 5 & \$3,507.45 & \$701.49 &
3.4\% \\
Asian American/Pacific Islander & 2 & \$1,772.00 & \$886.00 & 1.7\% \\
\end{longtable}

Upon comparison, the demographic trends between individuals using the
Housing Solutions Fund and the broader MCES population display similar
distributions. However, a notable discrepancy emerges, particularly in
the underrepresentation of Indigenous and Black individuals among users
of the Housing Solutions Fund when contrasted with the demographic
composition of the overall MCES population.

\begin{longtable}[]{@{}
  >{\raggedright\arraybackslash}p{(\columnwidth - 4\tabcolsep) * \real{0.4792}}
  >{\raggedleft\arraybackslash}p{(\columnwidth - 4\tabcolsep) * \real{0.2292}}
  >{\raggedright\arraybackslash}p{(\columnwidth - 4\tabcolsep) * \real{0.2917}}@{}}
\caption{Breakdown of The Overall MCES Population by
Race/Ethnicity}\tabularnewline
\toprule\noalign{}
\begin{minipage}[b]{\linewidth}\raggedright
Race/Ethnicity
\end{minipage} & \begin{minipage}[b]{\linewidth}\raggedleft
Number of Individuals
\end{minipage} & \begin{minipage}[b]{\linewidth}\raggedright
Percent of Total Population
\end{minipage} \\
\midrule\noalign{}
\endfirsthead
\toprule\noalign{}
\begin{minipage}[b]{\linewidth}\raggedright
Race/Ethnicity
\end{minipage} & \begin{minipage}[b]{\linewidth}\raggedleft
Number of Individuals
\end{minipage} & \begin{minipage}[b]{\linewidth}\raggedright
Percent of Total Population
\end{minipage} \\
\midrule\noalign{}
\endhead
\bottomrule\noalign{}
\endlastfoot
White & 3257 & 67.6\% \\
American Indian, Alaska Native, or Indigenous & 639 & 13.3\% \\
Multiple & 551 & 11.4\% \\
Black, African American, or African & 241 & 5.0\% \\
Asian American/Pacific Islander & 93 & 1.9\% \\
Hispanic/Latina/e/o & 40 & 0.8\% \\
\end{longtable}

\subsubsection{Conclusion}\label{conclusion}

\paragraph{Spending Distribution:}\label{spending-distribution}

Prevention Focus: The majority of funds are directed toward preventing
homelessness, predominantly through rent support. This aligns with
CPDI's objective of preventing homelessness proactively.

\paragraph{Funding Sources:}\label{funding-sources}

Primary Sources: The major contributors to the fund are the Affordable
Housing Trust Fund (AHTF), Wells Fargo (WF), and Emergency Food and
Shelter Grant (EFSG). AHTF and WF show substantial average spending per
individual, with EFSG displaying high average spending, albeit based on
a smaller sample size.

\paragraph{Demographic Analysis:}\label{demographic-analysis}

Ethnic Disparities: While the funding primarily benefits the White
demographic (84\%), it's notably higher than the 67\% representation of
Whites in the general unhoused population within MCES. This disparity
indicates potential inequities in fund allocation across minority
groups.

These insights highlight both the alignment of fund distribution with
prevention goals and the need for further exploration of equity in
funding distribution among various demographic groups within the
unhoused population.

\end{document}
